\documentclass[letterpaper,10pt]{article}

\usepackage{graphicx}                                        
\usepackage{amssymb}                                         
\usepackage{amsmath}                                         
\usepackage{amsthm}                                          

\usepackage{alltt}                                           
\usepackage{float}
\usepackage{color}
\usepackage{url}
\usepackage{hyperref}
\usepackage{balance}
\usepackage[TABBOTCAP, tight]{subfigure}
\usepackage{enumitem}
\usepackage{pstricks, pst-node}
\usepackage{listings}

\usepackage{geometry}
\geometry{textheight=8.5in, textwidth=6in}

%random comment

\newcommand{\cred}[1]{{\color{red}#1}}
\newcommand{\cblue}[1]{{\color{blue}#1}}

\newcommand{\toc}{\tableofcontents}

%\usepackage{hyperref}

\def\name{Hyun Song}



%% The following metadata will show up in the PDF properties
% \hypersetup{
%   colorlinks = false,
%   urlcolor = black,
%   pdfauthor = {\name},
%   pdfkeywords = {cs311 ``operating systems'' files filesystem I/O},
%   pdftitle = {CS 311 Project 1: UNIX File I/O},
%   pdfsubject = {CS 311 Project 1},
%   pdfpagemode = UseNone
% }

\parindent = 0.0 in
\parskip = 0.1 in

\begin{document}
Hyun Song
931 900 227

\section{Assignment 4 Write Up}
{\LARGE Design}\\
Looking at the assignment description from a few years ago, I know I have to make most of the modifications in the slob\_alloc(). Since the original slob.c uses first fit algorithm, I just have to change how it allocates by finding the smallest block but also that is big enough. The mofication necessary should be, minimal, as it just needs to iterate through each page, and check if it is the smallest block that is big enough. After that I have to set up to use my own syscalls that would return amount free and the amount claimed in an allocation request. I can use mmake menuconfig to change the configuration so that the kernel uses the slob allocator instead of the slab allocator.\\
\\



{\LARGE Version control Log}\\


commit 92b899713ac2d9e7edaefda6a56922gaa5db887\\
Author: Hyun Song <songh@oregonstate.edu>\\
Date:   Wed Nov 25 15:06:08 2015 -0800\\

    Syscalls actually work\\

commit f71062192b899713ac7daa09d29e32df7a6a73dd\\
Author: Hyun Song <songh@oregonstate.edu>\\
Date:   Mon Nov 23 19:16:41 2015 -0800\\

    should work now\\

commit b519c9f225afd35c025490568484be36925db7b4\\
Author: Hyun Song <songh@oregonstate.edu>\\
Date:   Mon Nov 23 19:07:35 2015 -0800\\

    should compile now\\

commit 7ea211d079cc242d9e7edaefda6a5fcf4bc91fd3\\
Author: Hyun Song <songh@oregonstate.edu>\\
Date:   Mon Nov 23 19:00:18 2015 -0800\\

    fixed a bunch of errors\

commit b2649e20939b8a699b5d8b4158f1b1ff22dc8bf9\\
Author: Hyun Song <songh@oregonstate.edu>\\
Date:   Sun Nov 22 18:20:28 2015 -0800\\

    added function to find best fit\\

commit 3640bccd6f641d3966f40dcc58dd6d601da53cf4\\
Author: Hyun Song <songh@oregonstate.edu>\\
Date:   Sun Nov 22 16:38:51 2015 -0800\\

    added syscalls\\

commit 7219659e2f3bce24a11961afb9ea5be507a379b4\\
Author: Hyun Song <songh@oregonstate.edu>\\
Date:   Sun Nov 22 15:08:07 2015 -0800\\

    init commit\\

commit a32039c3bce24a1199jo78a5be507a314be53da\\
Author: Hyun Song <songh@oregonstate.edu>\\
Date:   Sat  Nov 21 12:40:03 2015 -0800\\

    designing\\



{\LARGE Work Log}\\

Sat (11/21/15): Mostly consisted of researching and coming up with the design.\\

Sun (11/22/15): Did most of the implementation on slob\_alloc().\\

Mon (11/23/15): Just trying to get the kernel to run..\\

Tue-Wed (11/24~25/15): Getting the kernel to run properly and implemeenting the syscalls.\\

{\LARGE Questions}\\

\begin{enumerate}
\item The main purpose of this assignment was to modify the slob allocator so that it uses best fit algorithm instead of first fit algoritihm. The motivation behind this was because the first fit algorithm yields a high fragmentation percentage. Also, we needed to learn about the memory management layer, which takes care of the memory allocation requests.
\\
\item I personally approached this problem by first doing a lot of research and reading the chapters 6 and 12 in the Love text book. I then realized that the book does not show you how to implement your own system calls for Linux version of 3.14. So it took longer to figure out what to do to correctly implement the system calls. Other than that, I just refered to my design above to finish my project.
\\
\item I ensured that my solution was correct by using the system calls I implemented. I used the system calls on both the first fit slob allocator and the best fit slob allocator. The output from the system calls would indicate that there was still some fragmentation for the best fir algorithm, but still the amount that is free after allocation was a lot higher than the first fit algorithm.
\\
\item The most important thing I learned from this project is how the memory allocator works in general. I also learned about implementing my own system calsl by editing the system call table, and writing prototypes.\\ 


\end{enumerate}



%\section*{Source code:}

%\lstinputlisting{concurrency.c}
%\lstinputlisting{buffer.c}
%\lstinputlisting{buffer.h}

\end{document}
